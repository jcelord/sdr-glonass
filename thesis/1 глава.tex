\documentclass[a4paper,12pt]{report} %размер бумаги устанавливаем А4, шрифт 12пунктов
\usepackage[T2A]{fontenc}
\usepackage[utf8]{inputenc}%включаем свою кодировку: koi8-r или utf8 в UNIX, cp1251 в Windows
\usepackage[english,russian]{babel}%используем русский и английский языки с переносами
\usepackage{amssymb,amsfonts,amsmath,mathtext,cite,enumerate,float} %подключаем нужные пакеты расширений
\usepackage[dvips]{graphicx} %хотим вставлять в диплом рисунки?
\graphicspath{{images/}}%путь к рисункам

\makeatletter
\renewcommand{\@biblabel}[1]{#1.} % Заменяем библиографию с квадратных скобок на точку:
\makeatother

\usepackage{geometry} % Меняем поля страницы
\geometry{left=2cm}% левое поле
\geometry{right=1.5cm}% правое поле
\geometry{top=1cm}% верхнее поле
\geometry{bottom=2cm}% нижнее поле

\renewcommand{\theenumi}{\arabic{enumi}}% Меняем везде перечисления на цифра.цифра
\renewcommand{\labelenumi}{\arabic{enumi}}% Меняем везде перечисления на цифра.цифра
\renewcommand{\theenumii}{.\arabic{enumii}}% Меняем везде перечисления на цифра.цифра
\renewcommand{\labelenumii}{\arabic{enumi}.\arabic{enumii}.}% Меняем везде перечисления на цифра.цифра
\renewcommand{\theenumiii}{.\arabic{enumiii}}% Меняем везде перечисления на цифра.цифра
\renewcommand{\labelenumiii}{\arabic{enumi}.\arabic{enumii}.\arabic{enumiii}.}% Меняем везде перечисления на цифра.цифра

\begin{document}
%\input{DiplomProject-Title}% это титульный лист
%\tableofcontents % это оглавление, которое генерируется автоматически
\chapter{Обзор современных СРНС}
\section{СРНС}
Спутниковые радионавигационные системы связи(далее ~--- СРНС) ~--- навигационные системы, в которых в качестве маяков используются искусственные спутники Земли, несущие навигационную аппаратуру. 

[Определение, принцип работы, основные параметры, по которым сравнивают СРНС]
%\section{GPS}
[История GPS, характеристики]
%\section{ГЛОНАСС}
[История ГЛОНАСС, характеристики]
%\section{Beidou}
[История Beidou, характеристики, посмотреть презен. в списке лит. Ал. Бор.]
%\section{Galileo}
[История Galileo, характеристики]
\section{Проблема совместимости СРНС}
Из-за активного развития СРНС актуальной становится проблема бесконфликтного существования различных СРНС друг с другом. Рассмотренные СРНС работают в одном частотном диапазоне. 

[Написать про проблему ЭМС СРНС друг с другом и с другими системами, про спектр.-эфф. сигналы,сводная таблица с диап. частотами, написать, что L-диап кончается, написать про возможные пути решения(другие частотные полосы,...)]
\end{document}
